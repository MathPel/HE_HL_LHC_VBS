%% Main-Version Draft
\documentclass[11pt,epsf]{article}
%% -------------------------------
%% |          Packages           |
%% -------------------------------
 \usepackage{amsmath}
 \usepackage{graphicx}
 \usepackage[merge,numbers,compress]{natbib}
 \usepackage[T1]{fontenc}
 \usepackage{booktabs}
 \usepackage{xcolor} 
 \usepackage{xspace}
 \usepackage{dcolumn}
 \usepackage{hyperref}
 \usepackage{caption}
 \usepackage
 [subrefformat=parens,position=top,skip=-15pt,margin=15pt,justification=justified,singlelinecheck=false]
 {subcaption}

\setlength{\evensidemargin}{0cm}
\setlength{\oddsidemargin}{0cm}
\setlength{\topmargin}{0.00cm}
\setlength{\textwidth}{16.0cm}
\setlength{\textheight}{22.55cm}
\setlength{\headheight}{0cm}
\setlength{\headsep}{0cm}
\setlength{\voffset}{0cm}
\setlength{\paperheight}{27cm}

\renewcommand{\topfraction}{0.8}
\renewcommand{\bottomfraction}{0.5}
\renewcommand{\textfraction}{0.2}
\renewcommand{\floatpagefraction}{0.7}

\newcommand{\MP}[1]{{ {\color{blue}{ [MP: #1]}} }}

%% ---------------------------------
%% | ToDo Marker - only for draft! |
%% ---------------------------------
% Remove this section for final version!
\setlength{\marginparwidth}{20mm}

\newcommand{\margtodo}
{\marginpar{\textbf{\textcolor{red}{ToDo}}}{}}

\newcommand{\todo}[1]
{{\textbf{\textcolor{red}{(\margtodo{}#1)}}}{}}

\input{macros}

\begin{document}

\title{\hfill ~\\[-30mm]
\phantom{h} \hfill\mbox{\small PREPRINT}
\\[1cm]
\vspace{13mm}   \textbf{VBS for HE and HL LHC}}

\date{}
\author{
Ansgar Denner$^{1\,}$\footnote{E-mail:
  \texttt{ansgar.denner@physik.uni-wuerzburg.de}},
Mathieu Pellen$^{1\,}$\footnote{E-mail:
  \texttt{mathieu.pellen@physik.uni-wuerzburg.de}},
Michael Rauch
\\[9mm]
{\small\it
$^1$Universit\"at W\"urzburg, %
        Institut f\"ur Theoretische Physik und Astrophysik,} \\ %
{\small\it Emil-Hilb-Weg 22, \linebreak %
        97074 W\"urzburg, %
        Germany}\\[3mm]
}

\maketitle

\begin{abstract}
\noindent

\end{abstract}
\thispagestyle{empty}
\vfill
\newpage
\setcounter{page}{1}

\tableofcontents
\newpage


\section{Introduction}

The measurements of vector-boson scattering (VBS) processes are one of the most important in the future Large Hadron Collider (LHC) program.
In particular, given the foreseen experimental precision, very precise measurements will be performed.
This offers great opportunities to the probe the electroweak (EW) sector and its associated symmetry breaking mechanism
(see Refs.~\cite{Mangano:2016jyj,Goncalves:2017gzy,Jager:2017owh} for $100\TeV$ colliders studies).
Therefore it is of prime importance to estimate precisely theoretical predictions for the future operation of the LHC.
In this contribution, predictions for NLO QCD and EW corrections are provided for the LHC running in its high luminosity configuration (HL) and high energy (HE) configurations.
The HL set-up corresponds to a centre-of-mass energy of $14\TeV$ while the HE one is of $27\TeV$.
For both centre-of-energy the same type of event selections have been used.
These predictions represent important benchmarks as they indicates the expected rates and the corresponding theoretical uncertainties.
For QCD corrections, they are very important as they can significantly distort the shape of jet-related observables \cite{Jager:2006zc,Jager:2006cp,Bozzi:2007ur,Jager:2009xx,Jager:2011ms,Denner:2012dz,Rauch:2016pai,Biedermann:2017bss,Ballestrero:2018anz}.
In addition, the inclusion of NLO QCD corrections allow for more precise predictions as it reduces the theoretical uncertainties.
Concerning NLO EW, they have been showed to be very large for VBS processes \cite{Biedermann:2016yds} and even the dominating NLO contribution for same-sign WW scattering \cite{Biedermann:2017bss}.

The NLO QCD corrections have been obtained from \cite{Arnold:2008rz, Arnold:2011wj, Baglio:2014uba} in the so-called VBS approximation \cite{Ballestrero:2018anz}.
The NLO EW corrections have been obtained from MoCaNLO+Recola \cite{Bendavid:2018nar,Actis:2016mpe,Actis:2016mpe} in the full computation.
The differences between the full computation and the VBS approximated one have been found to be below the per-cent level \cite{Ballestrero:2018anz}.
This justifies the combination of this two types of corrections.
The QCD corrections have been computed for all possible VBS signatures while the EW ones have been restricted to the same-sign WW scattering.
This is justified given the fact that it constitutes a representative channel.
While the exact value of the corrections is expected to be different for other signatures, it is expected that their magnitudes and nature will be identical.

\section{Set-up}

The hadronic scattering processes are simulated at the LHC with a centre-of-mass energy $\sqrt s = 14 \TeV$ and $\sqrt s = 27 \TeV$.
    The NNNPDF~3.1 luxQED parton distribution functions~(PDFs)~\cite{Bertone:2017bme} with five massless flavours,\footnote{For the process considered, no bottom (anti-)quarks appear in the initial or final state at LO and NLO, as they would lead to top quarks, and not light jets, in the final state.} 
    NLO-QCD evolution, and a strong coupling constant $\alphas\left( \MZ \right) = 0.118$ are employed.\footnote{The corresponding identifier {\tt lhaid} in the program LHAPDF6~\cite{Buckley:2014ana} is 324900.}
    Initial-state collinear singularities are factorised according to the ${\overline{\rm MS}}$ scheme, consistently with what is done in the NNPDF set.

    For the rest of the input parameters, they have been chosen as in Ref.~\cite{Ballestrero:2018anz}.
    For the massive particles, the following masses and decay widths are used:
    %
    \begin{alignat}{2}
                      \Mt   &=  173.21\GeV,       & \quad \quad \quad \Gt &= 0 \GeV,  \nonumber \\
                    \MZOS &=  91.1876\GeV,      & \quad \quad \quad \GZOS &= 2.4952\GeV,  \nonumber \\
                    \MWOS &=  80.385\GeV,       & \GWOS &= 2.085\GeV,  \nonumber \\
                    M_{\rm H} &=  125.0\GeV,       &  \GH   &=  4.07 \times 10^{-3}\GeV.
    \end{alignat}
    %
    The measured on-shell (OS) values for the masses and widths of the W and Z bosons are converted into pole values for the gauge bosons ($V=\PW,\PZ$) according to Ref.~\cite{Bardin:1988xt},
    %
    \begin{equation}
    \begin{split}
            M_V &= \MVOS/\sqrt{1+(\GVOS/\MVOS)^2}\,, \\
       \Gamma_V &= \GVOS/\sqrt{1+(\GVOS/\MVOS)^2}.
    \end{split}
    \end{equation}
    %
    The EW coupling is renormalised in the $G_\mu$ scheme \cite{Denner:2000bj} according to 
    \begin{equation}
    \alpha =  \frac{\sqrt{2}}{\pi} G_{\mu} M_{\rm W}^2 \left(1-\frac{M_{\rm W}^2}{M_{\rm Z}^2} \right),
    \end{equation}
    with
    %
    \begin{equation}
        G_{\mu}    = 1.16637\times 10^{-5}\GeV^{-2},
    \end{equation}
    %
    and where $M_V^2$ corresponds to the real part of the squared pole mass.
    The numerical value of $\alpha$, corresponding to the choice of input parameters is
    %
    \begin{equation}
     1/\alpha = 132.3572\ldots\,.
    \end{equation}
    The Cabibbo--Kobayashi--Maskawa matrix is assumed to be diagonal, meaning that the mixing between different quark generations is neglected.
    The complex-mass scheme~\cite{Denner:1999gp,Denner:2005fg,Denner:2006ic} is used throughout to treat unstable intermediate particles in a gauge-invariant manner.

    The central value of the renormalisation and factorisation scales is set to 
    %
    \begin{equation}
    \label{eq:defscale}
     \mu_{\rm ren} = \mu_{\rm fac} = \sqrt{p_{\rm T, j_1}\, p_{\rm T, j_2}} .
    \end{equation}
    %
    The transverse momenta are the ones of the two hardest jets.
    This choice of scale has been shown to provide stable NLO-QCD predictions \cite{Denner:2012dz}.

    Following experimental measurements \cite{Aad:2014zda,Aaboud:2016ffv,Khachatryan:2014sta,CMS:2017adb} and prospect studies \cite{ATL-PHYS-PUB-2017-023}, the event selection used in the present study is:

    \begin{itemize}
        \item The two same-sign charged leptons are required to fulfil cuts on transverse momentum, rapidity, and separation in the rapidity--azimuthal-angle separation, 
            \begin{align}
            \label{cut:1}
             \ptsub{\Pl} >  20\GeV,\qquad |y_{\Pl}| < 4.0, \qquad \Delta R_{\Pl\Pl}> 0.3, \qquad m_{\Pl\Pl}>20\GeV.
            \end{align}
            %
        \item The total missing transverse momentum, computed from the vectorial sum of the transverse momenta of the two neutrinos, is required to be
            \begin{align}
            \label{cut:2}
              p_{\rm T, miss} >  40\GeV\,.
            \end{align}
            %
        \item QCD partons (light quarks and gluons) are clustered together using the anti-$k_T$ algorithm~\cite{Cacciari:2008gp}, with distance parameter $R=0.4$.
        We impose cuts on the jets' transverse momenta and rapidities,  
            \begin{align}
            \label{cut:3}
             \ptsub{\Pj} >  30\GeV, \qquad |y_\Pj| < 4.0, 
            \end{align}
            VBS cuts are applied on the two jets with largest transverse momentum, unless otherwise stated. In particular, we impose a cut on the 
             in\-vari\-ant mass of the di-jet system, as well as on the rapidity separation of the two jets and their separation from leptons,
            \begin{align}
            \label{cut:4}
             m_{\Pj \Pj} >  500\GeV,\qquad |\Delta y_{\Pj \Pj}| > 2.5, \qquad \Delta R_{\Pj\Pl} > 0.3 .
            \end{align}
            %
        \item Finally, the centrality of the leptons is enforced as defined in Ref.~\cite{ATL-PHYS-PUB-2017-023}:
            \begin{align}
            \label{cut:5}
             \zeta = \text{min}\left[\text{min}\left(y_{\Pl_1},y_{\Pl_2}\right) - \text{min}\left(y_{\Pj_1},y_{\Pj_2}\right), \text{max}\left(y_{\Pj_1},y_{\Pj_2}\right) - \text{max}\left(y_{\Pl_1},y_{\Pl_2}\right)\right]> 0 .
            \end{align}
            
        \item When EW corrections are computed, real photons and charged fermions are clustered together using the anti-$k_T$ algorithm with
            radius parameter $R=0.1$. In this case, leptons and quarks are understood as {\it dressed fermions}.
    \end{itemize}

\section{Discussions}

% \subsection{Standard Model}

In this section we focus on the discussion of Standard Model predictions for the HL- and HE- LHC.
This entails both QCD and EW corrections that have been combined together.

% The event selections used for the various signature is summed-up in Table XX.
% The scale used is ...
% The parton distribution functions used are...

For VBS processes EW corrections are particularly large \cite{Biedermann:2016yds} and are therefore of prime importance.
They originate from the exchange of massive gauge bosons in the virtual corrections.
They tend to grow negatively large in the high-energy limit due to Sudakov logarithms.
As shown in Ref.~\cite{Biedermann:2016yds}, large EW corrections are an intrinsic feature of VBS at the LHC.
While this study was based on the same sign $\PW$ channel, the demonstration has been further confirmed by the computation of EW corrections to the $\PW\PZ$ channel.

As these corrections are large and given the foreseen experimental precision, it is expected to actually measure such corrections.
Because they involve all the interactions of the EW sector, measuring them would constitute a non-trivial test of the SM.
On the left hand-side of Fig.~\ref{fig:ew}, the distribution in the invariant mass of the two leading jets is shown at LO and NLO EW for the process $\Pp\Pp \to \mu^+ \nu_{\mu} \Pe^+ \nu_{\Pe} \Pj\Pj$ at $14\TeV$.
In addition, the band describes the expected statistical experimental uncertainty for a HL-LHC collecting $3000\fb^-1$ and represents a relative variation of $\pm1/N_{\rm obs}$ where $N_{\rm obs}$ is the number of observed events in each bin.
It is thus clear that with the expected luminosity, one is expected to be experimentally sensitive to the VBS process but also to its EW corrections by themselves.

\begin{figure}
% \hspace{-2cm}
\includegraphics[width=.5\textwidth]{{{Figures/histogram_invariant_mass_mjj12_rauch}}}
\includegraphics[width=.5\textwidth]{{{Figures/histogram_invariant_mass_all}}}
% \vspace*{-1em}
\caption{
Left: Differential distributions in the invariant mass of the two jets in $\Pp\Pp\to\mu^+\nu_\mu\Pe^+\nu_{\Pe}\Pj\Pj$ $14\TeV$ including NLO EW corrections (upper panel) and relative NLO EW corrections (lower panel)
The yellow band describes the expected statistical experimental
uncertainty for a high-luminosity LHC collecting $3000\fb^{-1}$ and
represents a relative variation of $\pm 1/\sqrt{N_{\rm obs}}$ where $N_{\rm obs}$ is the number of observed events in each bin.
Right: Differential distributions in the invariant mass of the visible system ($\Pe^+\mu^+\Pj\Pj$) in $\Pp\Pp\to\mu^+\nu_\mu\Pe^+\nu_{\Pe}\Pj\Pj$ at $27\TeV$ including NLO EW corrections (upper panel) and relative NLO EW corrections (lower panel).
}
\label{fig:ew}
\end{figure}

In Fig.~\ref{fig:ew}, the distribution in the invariant mass of the visible system ($\Pe^+\mu^+\Pj\Pj$) at $27\TeV$ is shown.
As expected, the corrections are larger for higher centre-of-mass energy due to the higher representative scale of the process.
In the tail of the distribution where new physics could play an important role, the corrections are particularly large to reach almost $20\%$.
% A resummation of the Sudakov logarithms would be necessary for such a centre-of-mass energy.
Note that in the present predictions, the real radiation for massive gauge bosons is not taken into account.
The cross sections for the two centre-of-mass energy at LO (using full matrix element) and NLO EW are given in Table \ref{tab:EWxsec}.
In both cases the corrections are rather large in this fiducial volume.

\begin{table}
\begin{center}
\begin{tabular}
{|c||ccc|}
%
\hline
  & $\sigma^{\rm LO}$~[fb] &  $\sigma^{\rm NLO}_{\rm EW}$~[fb] & $\delta_{\rm EW}~[\%]$
\\
\hline
$14\TeV$ & $\phantom{1}1.4282(2)$ & $\phantom{1}1.213(5)$& $-15.1$ \\
\hline
$27\TeV$ & $\phantom{1}4.7848(5)$ & $\phantom{1}3.881(7)$& $-18.9$ 
\\
\hline
%
\end{tabular}
\end{center}
\caption{Cross sections at LO (order $\mathcal{O}\left(\alpha^6 \right)$) and NLO EW (order $\mathcal{O}\left(\alpha^7 \right)$) for $\Pp \Pp \to \mu^+ \nu_\mu \Pe^+ \nu_{\Pe} \Pj\Pj$ at both $14\TeV$ and $27\TeV$ at the LHC.
The relative EW corrections are given in per cent and the digits in parenthesis indicates the integration error.}
\label{tab:EWxsec}
\end{table}

% \subsection{Beyond the Standard Model}
% 
% \MP{I left the set-up section for completeness but we should probably omit it in the final version}


    
\section*{Acknowledgements}

AD and MP acknowledge financial support by the
German Federal Ministry for Education and Research (BMBF) under
contract no.~05H15WWCA1 and the German Research Foundation (DFG) under
reference number DE 623/6-1.

% \appendix


\bibliographystyle{utphys.bst}
\bibliography{vbs}
\end{document}
