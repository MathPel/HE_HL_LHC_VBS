%% Main-Version Draft
\documentclass[11pt,epsf]{article}
%% -------------------------------
%% |          Packages           |
%% -------------------------------
 \usepackage{amsmath}
 \usepackage{graphicx}
 \usepackage[merge,numbers,compress]{natbib}
 \usepackage[T1]{fontenc}
 \usepackage{booktabs}
 \usepackage{xcolor} 
 \usepackage{xspace}
 \usepackage{dcolumn}
 \usepackage{hyperref}
 \usepackage{caption}
 \usepackage
 [subrefformat=parens,position=top,skip=-15pt,margin=15pt,justification=justified,singlelinecheck=false]
 {subcaption}

\setlength{\evensidemargin}{0cm}
\setlength{\oddsidemargin}{0cm}
\setlength{\topmargin}{0.00cm}
\setlength{\textwidth}{16.0cm}
\setlength{\textheight}{22.55cm}
\setlength{\headheight}{0cm}
\setlength{\headsep}{0cm}
\setlength{\voffset}{0cm}
\setlength{\paperheight}{27cm}

\renewcommand{\topfraction}{0.8}
\renewcommand{\bottomfraction}{0.5}
\renewcommand{\textfraction}{0.2}
\renewcommand{\floatpagefraction}{0.7}

\newcommand{\MP}[1]{{ {\color{green}{ [MP: #1]}} }}

%% ---------------------------------
%% | ToDo Marker - only for draft! |
%% ---------------------------------
% Remove this section for final version!
\setlength{\marginparwidth}{20mm}

\newcommand{\margtodo}
{\marginpar{\textbf{\textcolor{red}{ToDo}}}{}}

\newcommand{\todo}[1]
{{\textbf{\textcolor{red}{(\margtodo{}#1)}}}{}}

\input{macros}

\begin{document}

\title{\hfill ~\\[-30mm]
\phantom{h} \hfill\mbox{\small PREPRINT}
\\[1cm]
\vspace{13mm}   \textbf{VBS for HE and HL LHC}}

\date{}
\author{
Ansgar Denner$^{1\,}$\footnote{E-mail:
  \texttt{ansgar.denner@physik.uni-wuerzburg.de}},
Mathieu Pellen$^{1\,}$\footnote{E-mail:
  \texttt{mathieu.pellen@physik.uni-wuerzburg.de}},
Michael Rauch
\\[9mm]
{\small\it
$^1$Universit\"at W\"urzburg, %
        Institut f\"ur Theoretische Physik und Astrophysik,} \\ %
{\small\it Emil-Hilb-Weg 22, \linebreak %
        97074 W\"urzburg, %
        Germany}\\[3mm]
}

\maketitle

\begin{abstract}
\noindent

\end{abstract}
\thispagestyle{empty}
\vfill
\newpage
\setcounter{page}{1}

\tableofcontents
\newpage


\section{Introduction}


\section{Set-up}

The hadronic scattering processes are simulated at the LHC with a centre-of-mass energy $\sqrt s = 14 \TeV$ and $\sqrt s = 27 \TeV$.
    The NNNPDF~3.1 luxQED parton distribution functions~(PDFs)~\cite{Bertone:2017bme} with five massless flavours,\footnote{For the process considered, no bottom (anti-)quarks appear in the initial or final state at LO and NLO, as they would lead to top quarks, and not light jets, in the final state.} 
    NLO-QCD evolution, and a strong coupling constant $\alphas\left( \MZ \right) = 0.118$ are employed.\footnote{The corresponding identifier {\tt lhaid} in the program LHAPDF6~\cite{Buckley:2014ana} is 324900.}
    Initial-state collinear singularities are factorised according to the ${\overline{\rm MS}}$ scheme, consistently with what is done in NNPDF.

    For the rest of the input parameters, they have been chosen as in Ref.~\cite{Ballestrero:2018anz}.
    For the massive particles, the following masses and decay widths are used:
    %
    \begin{alignat}{2}
                      \Mt   &=  173.21\GeV,       & \quad \quad \quad \Gt &= 0 \GeV,  \nonumber \\
                    \MZOS &=  91.1876\GeV,      & \quad \quad \quad \GZOS &= 2.4952\GeV,  \nonumber \\
                    \MWOS &=  80.385\GeV,       & \GWOS &= 2.085\GeV,  \nonumber \\
                    M_{\rm H} &=  125.0\GeV,       &  \GH   &=  4.07 \times 10^{-3}\GeV.
    \end{alignat}
    %
    The measured on-shell (OS) values for the masses and widths of the W and Z bosons are converted into pole values for the gauge bosons ($V=\PW,\PZ$) according to Ref.~\cite{Bardin:1988xt},
    %
    \begin{equation}
    \begin{split}
            M_V &= \MVOS/\sqrt{1+(\GVOS/\MVOS)^2}\,, \\
       \Gamma_V &= \GVOS/\sqrt{1+(\GVOS/\MVOS)^2}.
    \end{split}
    \end{equation}
    %
    The EW coupling is renormalised in the $G_\mu$ scheme \cite{Denner:2000bj} according to 
    \begin{equation}
    \alpha =  \frac{\sqrt{2}}{\pi} G_{\mu} M_{\rm W}^2 \left(1-\frac{M_{\rm W}^2}{M_{\rm Z}^2} \right),
    \end{equation}
    with
    %
    \begin{equation}
        G_{\mu}    = 1.16637\times 10^{-5}\GeV^{-2},
    \end{equation}
    %
    and where $M_V^2$ corresponds to the real part of the squared pole mass.
    The numerical value of $\alpha$, corresponding to the choice of input parameters is
    %
    \begin{equation}
     1/\alpha = 132.3572\ldots\,.
    \end{equation}
    The Cabibbo--Kobayashi--Maskawa matrix is assumed to be diagonal, meaning that the mixing between different quark generations is neglected.
    The complex-mass scheme~\cite{Denner:1999gp,Denner:2005fg,Denner:2006ic} is used throughout to treat unstable intermediate particles in a gauge-invariant manner.

    The central value of the renormalisation and factorisation scales is set to 
    %
    \begin{equation}
    \label{eq:defscale}
     \mu_{\rm ren} = \mu_{\rm fac} = \sqrt{p_{\rm T, j_1}\, p_{\rm T, j_2}}, 
    \end{equation}
    %
    defined via the transverse momenta of the two hardest jets (identified with the procedure outlined in the following), 
    event by event.
    This choice of scale has been shown to provide stable NLO-QCD predictions \cite{Denner:2012dz}.

    Following experimental measurements \cite{Aad:2014zda,Aaboud:2016ffv,Khachatryan:2014sta,CMS:2017adb} and prospect studies \cite{ATL-PHYS-PUB-2017-023}, the event selection used in the present study is:

    \begin{itemize}
        \item The two same-sign charged leptons are required to fulfil cuts on transverse momentum, rapidity, and separation in the rapidity--azimuthal-angle separation, 
            \begin{align}
            \label{cut:1}
             \ptsub{\Pl} >  20\GeV,\qquad |y_{\Pl}| < 4.0, \qquad \Delta R_{\Pl\Pl}> 0.3, \qquad m_{\Pl\Pl}>20\GeV.
            \end{align}
            %
        \item The total missing transverse momentum, computed from the vectorial sum of the transverse momenta of the two neutrinos, is required to be
            \begin{align}
            \label{cut:2}
              p_{\rm T, miss} >  40\GeV\,.
            \end{align}
            %
        \item QCD partons (light quarks and gluons) are clustered together using the anti-$k_T$ algorithm~\cite{Cacciari:2008gp}, with distance parameter $R=0.4$.
        We impose cuts on the jets' transverse momenta and rapidities,  
            \begin{align}
            \label{cut:3}
             \ptsub{\Pj} >  30\GeV, \qquad |y_\Pj| < 4.0, 
            \end{align}
            VBS cuts are applied on the two jets with largest transverse momentum, unless otherwise stated. In particular, we impose a cut on the 
             in\-vari\-ant mass of the di-jet system, as well as on the rapidity separation of the two jets and their separation from leptons,
            \begin{align}
            \label{cut:4}
             m_{\Pj \Pj} >  500\GeV,\qquad |\Delta y_{\Pj \Pj}| > 2.5, \qquad \Delta R_{\Pj\Pl} > 0.3 .
            \end{align}
            %
        \item Finally, the centrality of the leptons is enforced as defined in Ref.~\cite{ATL-PHYS-PUB-2017-023}:
            \begin{align}
            \label{cut:5}
             \zeta = \text{min}\left[\text{min}\left(y_{\Pl_1},y_{\Pl_2}\right) - \text{min}\left(y_{\Pj_1},y_{\Pj_2}\right), \text{max}\left(y_{\Pj_1},y_{\Pj_2}\right) - \text{max}\left(y_{\Pl_1},y_{\Pl_2}\right)\right]> 0 .
            \end{align}
            
        \item When EW corrections are computed, real photons and charged fermions are clustered together using the anti-$k_T$ algorithm with
            radius parameter $R=0.1$. In this case, leptons and quarks are understood as {\it dressed fermions}.
    \end{itemize}


\section*{Acknowledgements}

\appendix


\bibliographystyle{utphys.bst}
\bibliography{vbs}
\end{document}
